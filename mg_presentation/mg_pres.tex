%
\documentclass[8pt]{beamer}


\mode<presentation>
{
  \usetheme{Warsaw}
  \setbeamercovered{invisible}
}
\expandafter\def\expandafter\insertshorttitle\expandafter{%
 \insertshorttitle\hfill%
 \insertframenumber\,/\,\inserttotalframenumber}


%\usepackage{times}
\usepackage[english]{babel}

\usepackage[latin1]{inputenc}
\usepackage{graphicx}
\usepackage{media9}
    \usepackage[all]{xy}
    \usepackage{xypic}
  \usepackage{times}
    \usepackage{ulem}
    \usepackage[T1]{fontenc}
    \usepackage{amsfonts,amsmath,amssymb}
    \usepackage{hyperref}
    %\usepackage[all]{xy}
    \usepackage{amssymb,amsthm,amsxtra}
    %\usepackage[usenames]{color}
    \usepackage{amscd}
    \usepackage{amsthm}
    \usepackage{amsfonts}
    \usepackage{amssymb}
    \usepackage{mathrsfs}
    \usepackage{mathdots}
\usepackage{caption}
\usepackage{subcaption}
\usepackage{verbatim}

\newtheorem*{proposition}{Proposition}
\newtheorem*{prop}{Proposition}
\newtheorem*{remark}{Remark}
\newtheorem*{rem}{Remark}
\newtheorem*{question}{Question}


%\newtheorem*{example}{Example}
%\newtheorem*{theorem}{Theorem}
%\newtheorem*{definition}{Definition}
\newtheorem*{notation}{Notation}
%\newtheorem*{result}{Result}
%\newtheorem*{property}{}
%\newtheorem*{corollary}{Corollary}
%\newtheorem*{construction}{Construction}
%\newtheorem*{case}{Case}
\newtheorem*{conjecture}{Conjecture}
\newtheorem*{setting}{Setting}
\newtheorem*{flowchart}{Flowgausschart}

%\newtheorem{cor}[lemma]{Corollary}
%\newtheorem{exm}[lemma]{Example}
%\newtheorem{exc}[lemma]{Exercise}
%\newtheorem{conj}[lemma]{Conjecture}
%\newtheorem{rem}[lemma]{Remark}
%\newtheorem{conc}[lemma]{Conclusion}


\definecolor{DarkGreen}{rgb}{0,0.5,0}

\begin{document}


%%%%%%%%%%%%%%%%%%%%%% BEGIN COMMENT %%%%%%%%%%%%%%%%%%%%%%%%%%%%%%%%%
\begin{comment}

\title{Numerical Analysis in High Dimensions}
\author[]{ Nathan H., Andrew W., Andrew C., Ryan A., and Jeremiah C.\\ Dr. Pierre Gremaud, Rachael Gordon-Wright }

\begin{frame}
\titlepage
\end{frame}


\begin{frame}
\frametitle{Background}
\begin{itemize}
\item Uncertain parameters in equations  
\item A "solution" of the equation involves probability
\item Two options: Monte Carlo simulations or pdf of the function
\end{itemize}
\begin{figure}
\includegraphics[width=.8\linewidth]{CIex.jpg}
\end{figure}
\end{frame}


\begin{frame}
\frametitle{Outline}
\begin{itemize}
\item Description of sample problem
\vspace{4mm}
\item Our method: deriving and solving the pdf equation
\vspace{4mm}
\item "Old" method: Monte Carlo simulations
\vspace{4mm}
\item Discussion of results
\vspace{4mm}
\item Applying this method to other equations
\vspace{4mm}
\item Future work
\end{itemize}
\end{frame}

\begin{frame}
\frametitle{Sample Problem} 
\begin{itemize} 
\item Van der Pol equation
		\begin{itemize} 
		\item $\frac{d^2u}{dt} - \mu(1-u^2) \frac{du}{dt} + u = 0$ 
			\begin{itemize}
			\item $\mu$ is a parameter, describes strength of oscillations
			\end{itemize}
		\item Solve by splitting it up into two first-order ODEs
			\begin{itemize}
			\item $\frac{du}{dt} = v$
			\item $\frac{dv}{dt} - \mu(1-u^2) + u = 0$
			\end{itemize}
		\end{itemize} 
\item Initial conditions 
	\begin{itemize} 
	\item $u(0) = 1 + \eta$ 
	\item $v(0) = \zeta$ 
	\item $\eta, \zeta \sim \mathcal{N}(0,0.01)$
	\end{itemize}
\item Resolution of problem 
	\begin{itemize} 
	\item Stochastic initial conditions: Statistical approach 
	\item Probability Density Function: p(X) = P(x=X)
	\item Nonlinear system: Numerical approach
	\end{itemize}
\end{itemize}
\end{frame}
	
\begin{frame}

	\frametitle{Deriving the PDF equation for the Van der Pol Oscillator}
	\begin{itemize}

	\item We define the raw PDF function as:
	\begin{equation}
	\Pi = \delta(U-u(t))\delta(V-v(t)),
	\end{equation}
	where E[\textit{$\Pi$}] = p(U,V,t) (PDF function).
	\vspace{3 mm}
	
	\item We multiply \textit{$\frac{du}{dt}$} by \textit{$\frac{\partial\Pi}{\partial U}$}, and \textit{$\frac{dv}{dt}$} by \textit{$\frac{\partial\Pi}{\partial V}$} in order to get following equation:
	\begin{equation}
	\frac{\partial\Pi}{\partial t} + V\frac{\partial\Pi}{\partial U} + [\mu(1 - U^2)V-U]\frac{\partial\Pi}{\partial V} = 0
	\end{equation}

	\vspace{3 mm}
	\item Taking the expected value of both sides leads to the PDF equation:
	\begin{equation}
	\frac{\partial p}{\partial t} + V\frac{\partial p}{\partial U} + [\mu(1 - U^2)V-U]\frac{\partial p}{\partial V} = 0
	\end{equation}	

	\end{itemize}
\end{frame}

\begin{frame}
\frametitle{Results: PDE vs. Monte-Carlo}
\begin{figure}
\includegraphics[width=1\linewidth]{comparison_1.eps}
\caption{mu = 0.5}
\end{figure}
\end{frame}

\begin{frame}
\frametitle{Results: PDE vs. Monte-Carlo}
\begin{figure}
\includegraphics[width=1\linewidth]{comparison_2.eps}
\caption{mu = 1.5}
\end{figure}
\end{frame}

\begin{frame}
\frametitle{Results: PDE vs. Monte-Carlo}
\begin{figure}
\includegraphics[width=1\linewidth]{comparison_3.eps}
\caption{mu = 5}
\end{figure}
\end{frame}

\begin{frame}
\frametitle{Wave Equation}
\begin{itemize}
\item Standard Form
\begin{itemize}
\item $\frac{\partial^2 u}{\partial t^2}-c^2\frac{\partial^2 u}{\partial x^2} = 0$
\item u(x,t) = amplitude of wave
\item c = speed of propagation
\end{itemize}
\item First Order System
\begin{itemize}
\item Let $u_1 = \frac{\partial u}{\partial t}$ and $u_2 = \frac{\partial u}{\partial x}$
\item $\frac{\partial u_1}{\partial t} - c^2 \frac{\partial u_2}{\partial x} = 0$
\item $\frac{\partial u_2}{\partial t} - \frac{\partial u_1}{\partial x} = 0$
\end{itemize}
\item Analytical Solution to First Order PDEs (Advection Equations)
\begin{itemize}
\item If $\frac{\partial u}{\partial t} - c \frac{\partial u}{\partial x} = 0$
\item $u(x,t) = u_0(x-ct)$
\end{itemize}
\end{itemize}
\end{frame}

\begin{frame}
\frametitle{Sample Problem}
\begin{itemize}
\item System
\begin{itemize}
\item $\frac{\partial u_1}{\partial t} - c^2 \frac{\partial u_2}{\partial x} = 0$
\item $\frac{\partial u_2}{\partial t} - \frac{\partial u_1}{\partial x} = 0$
\item $c \sim \mathcal{U}(1-\epsilon,1+\epsilon)$
\end{itemize}
\item $2\pi$-periodic boundary conditions
\item Initial conditions
\begin{itemize}
\item $u_1(x,0) = sin(x) + \eta$
\item $u_2(x,0) = cos(x) + \zeta$
\item $\eta, \zeta \sim \mathcal{N}(0,1)$
\end{itemize}
\item Computing the CDF
\begin{itemize}
\item Analytical CDF
\begin{itemize}
\item Only for deterministic speeds $(\epsilon = 0)$
\end{itemize}
\item Monte Carlo simulation
\item Our derivation of the PDF/CDF
\begin{itemize}
\item Error from calculating expected values (closure rules)
\item Error from discretization
\end{itemize}
\end{itemize}
\end{itemize}
\end{frame}

\begin{frame}
\frametitle{Results}
\begin{figure}
\centering
\begin{subfigure}{.5\textwidth}
\centering
\includegraphics[width=1\linewidth]{cdf_ana_wave.eps}
\caption{Monte Carlo approximation to CDF}
\end{subfigure}%
\begin{subfigure}{.5\textwidth}
\centering
\includegraphics[width=1\linewidth]{cdf_num_wave.eps}
\caption{Numerical approximation to CDF}
\end{subfigure}
\end{figure}
\end{frame}

\begin{frame}
\frametitle{Future Steps and Applications}

\begin{columns}

\begin{column}{0.5 \textwidth}
\begin{itemize}
\setlength{\itemsep}{5pt}

\item Previously mentioned issues

\item Expand method to a broader range of problems: 
	\begin{itemize}
	\setlength{\itemsep}{2pt}
	\item Lorenz equations model atmospheric convection, and are very sensitive to initial conditions. 
		$$\left\{\begin{array}{l}
		\displaystyle{\frac{dx}{dt} = \sigma (y - x)} \\ \\
		\displaystyle{\frac{dy}{dt} = x(\rho - z) - y} \\ \\
		\displaystyle{\frac{dz}{dt} = xy - \beta z}\end{array}\right.$$
		
	\item Other ODEs 
	\item PDEs (wave and acoustic wave equations, specifically)	\end{itemize}
\end{itemize}

\end{column}

\begin{column}{0.5 \textwidth}
\begin{figure}

\includegraphics[width = 2in]{lorenz_attractor.png}
\caption{The Lorenz Butterfly}

\end{figure}
\end{column}

\end{columns}

\end{frame}

\begin{frame}

\frametitle{Method of Lines}

\begin{columns}

\begin{column}{0.5 \textwidth}
\begin{itemize}

\item Use semi-discretization (method of lines) to estimate the pdf of a stochastic partial differential equation.

	\begin{itemize} 
	\setlength{\itemsep}{2pt}
	\item $\displaystyle{\frac{du_i}{dt} = \mathfrak{F}(u_{i-1}, u_i, u_{i+1})}$
	\item Each differential equation represents a node of the discretization
	\item Number of dimensions increases quickly
	\end{itemize}

\end{itemize}

\end{column}

\begin{column}{0.5 \textwidth}
\begin{figure}

\includegraphics[width = 2in]{MOL.png}
\caption{Method of Lines for a heat problem}

\end{figure}
\end{column}
\end{columns}
\end{frame}

\begin{frame}
\frametitle{Acknowledgments}
\begin{itemize}
\item National Science Foundation grant numbers DMS-1063010 and DMS-0943855
\item National Security Agency grant number H98230-12-0299
\item Faculty mentor: Dr. Pierre Gremaud
\item Graduate mentor: Rachael Gordon-Wright
\end{itemize}
\end{frame}

\end{comment}
%%%%%%%%%%%%%%%%%%%%%%% END COMMENT %%%%%%%%%%%%%%%%%%%%%%%%%%%%%%%


\end{document}